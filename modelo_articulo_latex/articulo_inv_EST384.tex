\documentclass[11pt]{article}
\usepackage[letterpaper,margin=2cm]{geometry}
\usepackage[utf8]{inputenc}
\usepackage[spanish]{babel}
\usepackage[T1]{fontenc}
\usepackage{amssymb, amsmath, amsbsy, amsfonts}
\usepackage{upgreek}
\usepackage{graphicx}
\usepackage{multicol}
\usepackage{color}
\usepackage{caption}
\renewcommand{\figurename}{Fig.}
\renewcommand{\tablename}{Tab.}
\setlength{\columnsep}{5mm} 
\title{\textbf{Principales Factores de Violencia Contra las Mujeres}}
\author{Enrique Alejandro Laurel Cossio\\
\\Materia:Programacíon Estadistica II
\\Universidad Mayor de San Andres}
\date{11/07/2020}
\begin{document}
\maketitle
\begin{abstract}
\noindent Por medio de una medicion indirecta se cuntifico el momento magnetico de un iman a un nivel de confianza del 95\% mediante la aplicacion de la teoria de peqeuñas muestras que dio un resultado de: 
\[
R_{exp} = (18.17\pm 0.19)\left[ K\Omega\right] \hspace{0.5cm} ;N.C. = 95\%
\]
Donde los errores relativos para Z y M tiene los valores de:
\\
Palabras clave: Iman de ferrita, balanza analigica, regla comun de 30 [cm], teoria de peqeuñas muestras.
 \end{abstract}
\begin{center}
\textbf{Abstract}
\end{center}
\hspace{0.75cm} Acá viene el resumen en inglés \\
\indent \hspace{0.35cm}Keywords: Aca se pone los descriptores experimentales en inglés
%%%%%%
\begin{multicols}{2}
\section{\textbf{\textcolor{blue}{Introdución}}}
\noindent 1.- ¿Porque un iman permanente tiene las propiedades de atraer o repeler objetos metalicos?\\
2.-¿Cual es la diferencia entre imanes de Ferrita y Neodimio?
\section{\textbf{\textcolor{blue}{Objetivos}}}
\subsection{\textcolor{magenta}{Objetivo general}}
\noindent Estudio del momento magnetico de un iman de Ferrita como funcion de la distancia de separacion Z y la influencia de la fuerza gravitacional.
\subsection{\textbf{\textcolor{magenta}{Objetivo específico}}}
\noindent Determinar el momento magnetico de un iman de Ferrita po el metodo indirecto a un nivel de confianza del 95\% mas erroe relativo pocentual.
\section{\textbf{\textcolor{blue}{Marco teórico}}}
\noindent El momento magnetico de un iman es una cantidad que determina la fuerza que el iman puede ejercer sobre corrientes electricas y el torque que un campo magnetico ejerce sobre ellas.Cuando los imanes tiene la polaridad opuesta, estos se atraen de forma que cuando se separan la fuerza magnetica netre estos los atrae y a medida que la distancia de separacion aumenta la fuerza magnetica se debilita y la fuerza gravitacional se hace intensiva hasta que la accion de la  fuerza,peso del iman este cae en ese instante se rompe el equilibrio entre la fuerza magnetica ya fuerza gravitacional entonces la expresion matematica para el momento magnetico del iman es:
Los comandos LateX permiten obtener una fórmula de la forma: 
\begin{equation}
\textup{\LARGE $\sigma$}{\hspace{-0.1cm}N-1} \cdot \frac{\partial x}{\partial t} = \sum{j=1}^{n}z_{j}
\end{equation}
\begin{equation}
\textup{\LARGE $\sigma$}_{\hspace{-0.1cm}N-1}
\end{equation}
\begin{equation}
\textup{\large $\Omega$}_{\hspace{-0.03cm}N-1}=1
\end{equation}

\section{\textbf{\textcolor{blue}{Marco experimental}}}
\subsection{\textbf{\textcolor{magenta}{Introducción}}}
\noindent Acá viene una breve descripción experimental con los instrumentos a ser utilizados en la experiencia con el circuito eléctrico que acompaña, si es el caso.
\begin{center}
%\includegraphics[scale=0.7]{resistencia.jpg}
\captionof{figure}{Se observa códigos de colores para la lectura correcta en una resistencia eléctrica.\label{fig01}}
\end{center}
\noindent La figura \ref{fig01} muestra un esquema para la lectura por código de colores a una cierta resistencia eléctrica.
\subsection{\textbf{\textcolor{magenta}{Datos Experimentales}}}
\noindent En esta sección va la Tabla de valores medidos u obtenidos mediante un instrumento de medición, por ejemplo:

\begin{center}
  \begin{tabular}{|c||c|c|}   \hline
 N&V[v] & I[mA] \\[0,1cm] 
            \hline \hline
            1 & & \\ \hline
            2 & &  \\ \hline
            3 & &  \\ \hline
            4 & &  \\ \hline
            5 & &  \\ \hline
            6 & &  \\ \hline
            7 & &  \\ \hline
            8 & &  \\ \hline
            9 & &  \\ \hline
            10& &  \\ \hline
            11& &  \\ \hline
            12& &  \\ \hline
            13& &  \\ \hline
            14& &  \\ \hline
            15& &  \\ \hline
\end{tabular}
 \end{center}
\captionof{table}{Se muestra en la tabla los valores experimentales medidos a partir de un multímetro digital y un miliamperímetro ananlogico.}
\section{\textbf{\textcolor{blue}{Resultados y análisis}}}
\noindent aslklneopnoopnfgeo.....
\section{\textbf{\textcolor{blue}{Conclusiones}}}
\noindent ioruiownfvoawpoepao....
\begin{thebibliography}{10}
\bibitem{1} D. C. Baird (1995). Experimentación: Una introducción a la teoría de mediciones y al diseño de experimentos (2da Ed.) Mexico: Prentice-Hall Hispanoamericana.
\bibitem{2} Alvarez, A. C. y Huayta, E. C. (2008). Medidas y Errores (3ra Ed.) La Paz - Bolivia: Catacora.
\end{thebibliography}
\end{multicols}
\end{document}